\documentclass[11pt]{article}

\usepackage{fontspec}
% Libertine
\setmainfont[Ligatures={Common,TeX}]{Linux Libertine O}
\setsansfont{Linux Biolinum O}
\setmonofont{Inconsolata}

\usepackage{minted}
\newminted{julia}{}

\usepackage{geometry}
\geometry{
	xetex,
	textwidth=410pt
}

\hyphenpenalty 300

\title{Gaston \\[0.8cm] \large A gnuplot library for Julia \\[0.8cm] v. 0.2}
\author{M. Bazdresch}
\date{\today}

\newcommand{\cmd}[1]{\texttt{#1}}

\begin{document}
\maketitle

\textbf{Please note:} Gaston is currently under development, and all functions
and definitions are subject to change from one version to the next, as we
figure out the best way to organize the code. Gaston has been tested on Linux
(Ubuntu 10.04 and Arch), with gnuplot 4.6 and WxWindows. It only supports the
wxt terminal.

\section{Introduction}

Gaston provides a way to plot scientific data using the Julia programming
language. It accomplishes this by harnessing gnuplot, a versatile and
time-tested plotting utility. Gaston also relies on gnuplot for interacting
with plots (zooming and rotating a plot with the mouse, for instance).

The primary purpose of Gaston is to provide easy-to-use functions to quickly
and conveniently plot the most common kinds of scientific and numeric data. It
is concerned only with screen output (although we plan to support printing to a
file in a future version), and provides only for the most common
plot configurations. Tweaking a plot to produce a specific look or producing
publication-quality graphics are outside its scope.

\section{Installation}

To use Gaston, follow this procedure:

\begin{enumerate}

\item Save the files \cmd{gaston.jl}, \cmd{gastonini.jl}, \cmd{gastonaux.jl}
and \cmd{gastondemo.jl} somewhere convenient. Then, you may \cmd{cd} to that
directory and start julia there, or do

\begin{juliacode}
push(LOAD_PATH, "/path/to/gaston/jl")
\end{juliacode}

Then, load the program with

\begin{juliacode}
load("gaston.jl")
\end{juliacode}

\item To run a demo, do

\begin{juliacode}
load("gastondemo.jl")
demo()
\end{juliacode}

This will create a series of figures that illustrate the current capabilities
of the program. The same file may also serve as a guide on how to create
different types of plots.
\end{enumerate}

\section{Plotting}

In future versions, Gaston will offer plotting functions similar to Octave or
Matlab's -- what we call ``high-level'' functions. At the moment, it offers a
``mid-level'' set of functions that require that plots be created step-by-step.
Before delving into that process, some definitions are in order.

\subsection{Definitions}

A \textbf{figure} is an independent window, which contains a set of axes, on
which one or more \textbf{curves} are plotted. A figure may contain
\textbf{labels} (for instance, on each coordinate axis), a \textbf{title}, and a
\textbf{legend box} which identifies each curve. Gaston supports having any
number of figures open at the same time; however, gnuplot requires that only
one figure is able to offer mouse interactivity at a given time. Each figure is
identified by a unique \textbf{handle}. Handles are natural numbers.

A curve is defined by a set of coordinates. Two-dimensional curves have
\textbf{x} and \textbf{y} coordinates; in three dimensions, an additional
\textbf{z} coordinate must be specified. A curve also has a plotting
configuration (for instance, it may have a \textbf{linestyle}, a
\textbf{linewidth}, a \textbf{linecolor}, etc), which define how the curve
is to be plotted.

Please note that there is a single command to plot, which is \cmd{plot()}.
According to the type of coordinates and plotstyle, it will figure out how to
plot.

\subsection{2-D plotting}

Plotting proceeds in steps:
\begin{enumerate}
	\item Create or select a figure with \cmd{figure(i)}, where \cmd{i} is a
		positive integer.
	\item Add a curve (a set of coordinates plus a plot configuration), with
		\cmd{addcoords(x,y,conf)}. Here, \cmd{x} and \cmd{y} are vectors, and
		\cmd{conf} configures the plot and line styles, markers, legend, color,
		etc. Repeat this step for each curve you wish to include
		in the figure.
	\item Add a configuration for the entire figure (axis), with
		\cmd{addconf(conf)}, where \cmd{conf} contains the figure
		configuration.
	\item Issue the \cmd{plot()} command.
\end{enumerate}

A curve configuration is created as follows:
\begin{enumerate}
	\item Create a default configuration with \cmd{c = Curve\_conf()}.
	\item \cmd{c} is a structure, each of whose fields controls one aspect of
		the curve's configuration. These fields may be set individually.
		Available fields are: \cmd{legend}, \cmd{plotstyle}, \cmd{color},
		\cmd{marker}, \cmd{linewidth}, and \cmd{pointsize}. For instance, to
		change a curve's color, do\footnote{Directly changing a structure's
		fields, instead of using setter functions, may be seen as non-idiomatic
		or inelegant. In this case we have decided to do the simplest thing
		that might possibly work.}

\begin{juliacode}
c.color = "blue"
\end{juliacode}

	See gnuplot's documentation to see the range of valid options.

\end{enumerate}

A figure (axis) configuration is created as follows:
\begin{enumerate}
	\item Create a default configuration with \cmd{a = Axes\_conf()}
	\item Just as in the case of a curve configuration, \cmd{a} is a structure
		whose fields may be modified. Available fields are: \cmd{title},
		\cmd{xlabel}, \cmd{ylabel}, \cmd{zlabel}, \cmd{box}, \cmd{axis}.
\end{enumerate}

Several rules apply:
\begin{itemize}
	\item You can create as many figures, each with as many curves, as desired.
	\item Generally, if you don't provide some of the data, it will be
		inferred. For example, calling \cmd{addcoords} with a single vector
		\cmd{y}
		will assume the \cmd{x} coordinate is \cmd{1:length(y)} and set up the
		default plot configuration.
	\item If you call \cmd{addcoords} with matrix arguments, each column will
		be interpreted as a different plot.
	\item Calling \cmd{addcoords()} will create a new figure if none have been
		created yet.
	\item Calling \cmd{plot()} without an axis configuration will just use one
		by default.
	\item Gnuplot only provides mouse interaction support for the current
		figure. To use the mouse in a previously created figure \cmd{i}, just
		issue command \cmd{figure(i)}. This will also bring the figure to the
		front.
\end{itemize}

\subsubsection{2-D plotting styles}

Besides simply plotting a set of \cmd{x} and \cmd{y} coordinates, Gaston
supports other kinds of plots.

\textbf{Error bars and lines.} To add error bars or lines, just call
\cmd{addcoords()} with one or two extra coordinates, and configure the
plotstyle accordingly.

\textbf{Histograms.} To plot the histogram of a data vector \cmd{data} with
\cmd{b} bins, first use the auxiliary function \cmd{(x,y) = histdata(data,b)}
to create \cmd{x} and \cmd{y} coordinates that may be plotted with the
\cmd{boxes} plotstyle.

\subsection{3-D plotting}

The same rules apply, except that \cmd{addcoords()} should be called as
\cmd{addcoords(x,y,Z)}, where \cmd{Z} is a matrix whose element \cmd{j,k}
corresponds to some function of \cmd{x[j],y[k]}.

For convenience, a function \cmd{Z = meshgrid(x,y,f)} is provided. Called with
\cmd{x}, \cmd{y} coordinates and a function \cmd{f}, it will return a matrix
that may be used to plot \cmd{f}.

\subsection{Image plotting}

Two image plotstyles are supported: \cmd{image} and \cmd{rbgimage}. At the
moment Gaston requires that a figure contains only one image (and no other
curves). This restriction will be removed in future versions.

\textbf{Scalar image.} To plot a matrix \cmd{Z} as a figure, use
\cmd{addcoords([],[],Z)} with empty \cmd{x}, \cmd{y} coordinates and \cmd{Z} as
third argument, and set the plotstyle to \cmd{image}. Element \cmd{(j,k)} of
\cmd{Z} corresponds to element \cmd{(j,k)} in the plot, and its value determines
the color at that point.

\textbf{RGB image.} To plot an RGB image, the matrix \cmd{Z} must be three
dimensional. The first and second dimensions correspond to the value at each
point in the image.  The third dimension specifies red, green and blue values
at each point.  Finally, set the plotstyle to \cmd{rgbimage}.

\end{document}
